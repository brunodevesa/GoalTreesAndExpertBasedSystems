\documentclass[a4paper,12pt,notitlepage,twoside]{article}

% ------------ ativar os packages necessarios ----------

\usepackage[latin1]{inputenc} 		% para poder utilizar os caracteres tipicamente portugueses [�,~,^,...] 
\usepackage[portuguese]{babel}		% para escrever em Portugu�s de Portugal
\usepackage[T1]{fontenc}			% permite usar a hifeniza��o correcta em portugu�s

% Math
\usepackage{amssymb,amsbsy,amsmath}   % para utilizar matem�tica
\usepackage{amsfonts} %para poder fazer o R, Z etc \mathbb{R}	\mathbb{Z} 

% Figures
\usepackage{graphicx,color}			% para introduzir figuras
%\usepackage[pdftex,final]{graphicx}      % para introduzir figuras
%\usepackage[pdftex]{color}         % para introduzir cor
%\usepackage{wrapfig}						% permite colocar texto ao lado de figuras
\usepackage{floatflt}
\usepackage{subfigure, subfloat}
%\usepackage{subfig}				% para colocar varias imagens tal como o subfigure package mas para poder fazer figuras quebradas
\usepackage[small]{caption}        % legendas das figuras 

% hyperlinks
\usepackage{url}
\usepackage{hyperref}

% Index
\usepackage{makeidx,showidx} % para criar indice remissivo

% Outros packages
\usepackage{enumerate}             % para poder utilizar items diferentes da numera��o �rabe com o ambiente {enumerate}

\usepackage{longtable }           %supertabular
\usepackage{lscape}
\usepackage{array}
\usepackage{multirow}


%--------- cabe�alhos e rodap�s nas p�ginas --------
\usepackage{fancyhdr}
\pagestyle{fancy}
			\lhead{ESCOLA-IPP}
			\chead{ }
			\rhead{2015/2016}
			\lfoot{\hrule \vspace*{5pt}  Trabalho de Pr�tico de [COMPLETAR]}
			\cfoot{}
			\rfoot{\hrule \vspace*{5pt} \thepage}

% --------------- bibliography style -------------
\usepackage{natbib}
\bibliographystyle{spbasic}
%\bibliographystyle{plain}  %coloca as ref com numeracao[1],[2],...
%\bibliographystyle{ifac}						% estilo de bibliografia tipo IFAC
%\usepackage[dcucite]{harvard}					% usado qdo \bibliographystyle{ifac}
%\bibliographystyle{IEEEtranSN}		            % estilo da bibliografia tipo IEEE



% ------------- fun��es matem�ticas em portugues -----------

\def \sin {\textnormal{sen}} 
\def \tan {\textnormal{tg}}
\def \arcsin {\textnormal{arcsen}} 
\def \arctan {\textnormal{arctg}}
\def \cot {\textnormal{cotg}}
\def \csc {\textnormal{cosec}}


\newcommand{\cis}{\textnormal{cis}}
\newcommand{\sen}{\textnormal{sen}}
\newcommand{\neper}{\textnormal{e}}
\newcommand{\dx}{\,\,\textnormal{d}x}
\newcommand{\arccotg}{\textnormal{arccotg}}
\newcommand{\arcsec}{\textnormal{arcsec}}


% --------------- bibliography style -------------
\usepackage{natbib}
%\bibliographystyle{spbasic}
%\bibliographystyle{ifac}						% estilo de bibliografia tipo IFAC
%\usepackage[dcucite]{harvard}					% usado qdo \bibliographystyle{ifac}
%\bibliographystyle{IEEEtranSN}		            % estilo da bibliografia tipo IEEE

% -------------------------------------------------------------

\begin{document}

%--------------------------------------------------


\begin{titlepage}%

	\begin{center} {\includegraphics[width=5cm]{Figuras/Logo}}\\
	\end{center}
	 \vspace*{3cm}
	
	
	
  \begin{center}
  \huge{\textbf{Unidade Curricular [ALTERAR] }}\\
  \vspace{2cm}
  \Huge{\textbf{Trabalho Pr�tico}}\\
  \vspace{2cm}
  \normalsize{Elaborado por:} \\
  \large{[PREENCHER n�mero e nome do aluno]}\\
  \vspace{0.5cm}
  \large{Curso: [COMPLETAR]}\\
  \vspace{1cm}
  \normalsize{Docente:} \large{Eliana Costa e Silva -- eos@estgf.ipp.pt}
  	\end{center}  
		
  

\vfill 
	\begin{center}
  Felgueiras, ?? de abril de 2015
  	\end{center}  
        
\end{titlepage}%
\newpage

%--------------------------------------------------
\tableofcontents 	% �ndice
\newpage

\listoffigures   	% Lista de figuras

%\newpage
%\listoftables   	% Lista de tabelas
\
%--------------------------------------------------


%Introducao

\section{Introdu��o}

O trabalho dever� ter uma parte de introdu��o que explique os objetivos, o contexto do trabalho e os problemas colocados, uma parte de desenvolvimento (maioritariamente pr�tica) que ser� constitu�da pelas respostas a cada problema apresentado no enunciado, uma conclus�o e a bibliografia. 

INCLUIR A INTRODU��O AQUI!!!!!!!!!

Exemplo para obrigar a aparecer a bibliografia:
 Neste trabalho foram consultados \cite{Rosen2012}, \cite{Lipschutz2013}, \cite{Gersting2007}, .... OUTROS [MUDAR AS REFERENCIAS]. 

\subsection{Objetivos}
O presente trabalho tem como objetivos aprender/desenvolver... COMPLETAR

\subsection{Estrutura}
DESCREVER POR PALAVRAS O �NDICE DO VOSSO TRABALHO

Este trabalho est� estruturado em ??? cap�tulos/sec��es, sendo o primeiro cap�tulo uma pequena introdu��o em que se refere o contexto em que surgiu este trabalho, os objetivos que  se pretende atingir e a estrutura do mesmo. 
No segundo cap�tulo � feito isto assim assim e no terceiro � feito assim assado... (etc). 
O �ltimo cap�tulo apresenta as conclus�es e por fim � feita uma refer�ncia � bibliografia consultada.

 %include insere em nova p�gina
% ------------------------------------------------------------------------------
\section{Problema 1 [ALTERAR]}



AQUI VEM O ..... 
% ------------------------------------------------------------------------------
\section{Projeto 2 [Alterar]}



AQUI VEM O PROJETO 2 
% ------------------------------------------------------------------------------
% ------------------------------------------------------------------------------
%Conclusao do trabalho
\newpage
\section{Conclus�o}
Escrever a conclus�o do trabalho.


%input insere na mesma p�gina
% ------------------------------------------------------------------------------

\newpage
% + + + + + + + + + + + + + + + + + + + + + + + + + + + + + + + + + + + + + + + + + + + + 
\bibliography{Bibliografia}

\end{document}
