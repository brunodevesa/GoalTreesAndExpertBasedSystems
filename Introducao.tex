%Introducao

\section{Introdu��o}

O trabalho dever� ter uma parte de introdu��o que explique os objetivos, o contexto do trabalho e os problemas colocados, uma parte de desenvolvimento (maioritariamente pr�tica) que ser� constitu�da pelas respostas a cada problema apresentado no enunciado, uma conclus�o e a bibliografia. 

INCLUIR A INTRODU��O AQUI!!!!!!!!!

Exemplo para obrigar a aparecer a bibliografia:
 Neste trabalho foram consultados \cite{Rosen2012}, \cite{Lipschutz2013}, \cite{Gersting2007}, .... OUTROS [MUDAR AS REFERENCIAS]. 

\subsection{Objetivos}
O presente trabalho tem como objetivos aprender/desenvolver... COMPLETAR

\subsection{Estrutura}
DESCREVER POR PALAVRAS O �NDICE DO VOSSO TRABALHO

Este trabalho est� estruturado em ??? cap�tulos/sec��es, sendo o primeiro cap�tulo uma pequena introdu��o em que se refere o contexto em que surgiu este trabalho, os objetivos que  se pretende atingir e a estrutura do mesmo. 
No segundo cap�tulo � feito isto assim assim e no terceiro � feito assim assado... (etc). 
O �ltimo cap�tulo apresenta as conclus�es e por fim � feita uma refer�ncia � bibliografia consultada.

